\documentclass[12pt]{article}

\usepackage{fancyhdr}
\usepackage{extramarks}
\usepackage{amsmath}
\usepackage{amsthm}
\usepackage{amsfonts}
\usepackage{bm}
\usepackage{tikz}
\usepackage{graphicx}
\usepackage{float}
\usepackage{enumerate}
\usepackage{hyperref}

\usetikzlibrary{positioning}

%
% Basic Document Settings
%
\hypersetup{
    colorlinks=true,
    linkcolor=cyan,
    filecolor=magenta,      
    urlcolor=blue,
    pdftitle={Jaydev Singh Rao (19147): Planar VTOL Design Study}
    }

\topmargin=-0.45in
\evensidemargin=0in
\oddsidemargin=0in
\textwidth=6.5in
\textheight=9.0in
\headsep=0.25in

\linespread{1.1}

\pagestyle{fancy}
\lhead{\tiny{\hmwkAuthorName \space \hmwkAuthorRoll}}
\rhead{\tiny{\hmwkClass \ (\projecttitle)}}
\cfoot{\tiny\thepage}

\renewcommand\headrulewidth{0.4pt}
\renewcommand\footrulewidth{0.4pt}

\setlength\parindent{0pt}

\setcounter{secnumdepth}{0}
\newcounter{objectivecounter}
\setcounter{objectivecounter}{1}
\nobreak\extramarks{Objective \arabic{objectivecounter}}{}\nobreak{}

%
% Homework Problem Environment
%
% This environment takes an optional argument. When given, it will adjust the
% problem counter. This is useful for when the problems given for your
% assignment aren't sequential. See the last 3 problems of this template for an
% example.
%
\newenvironment{Objective}[1]{
    \section{\arabic{objectivecounter}. #1}
    \stepcounter{objectivecounter}
}{}

\newcommand{\projecttitle}{Planar VTOL System}
\newcommand{\hmwkTitle}{Design Study}
\newcommand{\hmwkClass}{ECS 323: Control Systems}
\newcommand{\hmwkAuthorName}{Jaydev Singh Rao}
\newcommand{\hmwkAuthorRoll}{19147}

%
% Title Page
%

\title{
    \vspace{1in}
    \hrule
    \vspace{0.25in}
    \textmd{\textbf{\Huge \hmwkClass}}\\
    \vspace{1in}
    \textmd{\textbf{\Huge \projecttitle}} \\
    \vspace{0.1in}\normalsize{\textmd{\LARGE \hmwkTitle}}\\
    \vspace{1in}
    \normalsize\vspace{0.1in}\LARGE{Submitted by: \hmwkAuthorName}\\
    \normalsize\vspace{0.1in}\LARGE{Roll No. : \hmwkAuthorRoll}
}

\author{}
\date{}

%
% Various Helper Commands
%

% For derivatives
\newcommand{\de}[2]{\cfrac{\mathrm{d}#1}{\mathrm{d}#2}}

% For partial derivatives
\newcommand{\pde}[2]{\cfrac{\partial #1}{\partial #2}}

%integral
\newcommand{\integrate}[4]{\displaystyle \int_{#3}^{#4} #1 \mathrm{d}#2}

% Integral dx
\newcommand{\dx}{\mathrm{d}x}

% Alias for the Solution section header
\newcommand{\solution}{\textbf{\large Solution}}

% Centered figure
\newcommand{\centerfig}[4][0.35]{
    \begin{figure}[H]
        \centering
        \fbox{\includegraphics[scale=#1]{#2}}
        \caption{#3}
        \label{fig:#4}
    \end{figure}
    }

\begin{document}

    \maketitle
    \hrule
    \pagebreak
    \tableofcontents
    \pagebreak

    \begin{Objective}{Design Study Description}
        In this design study we need to design a control system for the given planar VTOL system with the parameters:
        \begin{itemize}
            \item $M_c=2 \text{ kg}$
            \item $J_c = 0.009 \text{ kg m$^2$}$
            \item $m_l = 0.3 \text{ kg}$
            \item $m_r = 0.3 \text{ kg}$
            \item $d = 0.28 \text{ m}$
            \item $\mu = 0.21 \text{ kg s$^{-1}$}$
        \end{itemize}
        \centerfig[0.5]{./planar_vtol_diagram.png}{Planar VTOL System}{vtol_fig}
    \end{Objective}

    \pagebreak

    \begin{Objective}{Kinetic Energy}
        The postions of the various components of the VTOL are given by:
        \[ 
            \begin{split}
                \mathbf{p_c} & = (z_v,\ h) \\
                \mathbf{p_l} & = (z_v - d \cos\theta,\ h - d \sin\theta) \\
                \mathbf{p_r} & = (z_v + d \cos\theta,\ h + d \sin\theta)
            \end{split}    
        \]
        So, the velocities can be written as:
        \[ 
            \begin{split}
                \mathbf{v_c} & = (\dot z_v,\ \dot h) \\
                \mathbf{v_l} & = (\dot z_v + d\dot \theta \sin\theta,\ \dot h - d \dot \theta \cos\theta) \\
                \mathbf{v_r} & = (\dot z_v - d \dot \theta \sin \theta,\ \dot h + d \dot \theta \cos \theta)
            \end{split}    
        \]

        Kinetic energy of the centerpod is given by:
        \begin{equation}
            \label{KE_pod}
            K_{pod} = \frac12 m_c \mathbf{v}_c^T \mathbf{v}_c + \frac12 \bm{\omega}_c^TJ_c \bm{\omega}_c = \frac12 m_c (\dot z_v^2 + \dot h^2) + \frac12 J_c \dot \theta ^2
        \end{equation}

        Kinetic energy of the left and right rotors is given by:
        \begin{equation}
            \label{KE_rotors}
            \begin{split}
                K_{rotors} & = \frac12 m_l \mathbf{v}_l^T \mathbf{v}_l + \frac12 m_r \mathbf{v}_r^T \mathbf{v}_r \\
                    & = \frac12 m_l (\dot z_v + d\dot \theta \sin\theta)^2 + \frac12 m_l (\dot h - d \dot \theta \cos\theta)^2 \\
                    & \quad + \frac12 m_r (\dot z_v - d\dot \theta \sin\theta)^2 + \frac12 m_l (\dot h + d \dot \theta \cos\theta)^2 \\
                    & = \frac12 (m_l + m_r) (\dot z_v^2 + \dot h^2) + \frac12 (m_l + m_r) d^2 \dot \theta^2 \\
                    & \quad + (m_l-m_r)(\dot z_v \sin \theta - \dot h \cos \theta) d\dot \theta
            \end{split}
        \end{equation}

        Now, the total kinetic energy of the VTOL will be given by the sum of \ref{KE_pod} and \ref{KE_rotors}:
        \begin{equation}
            \label{KE_vtol}
            K_{V} = K_{pod} + K_{rotors}
        \end{equation}
    \end{Objective}

\end{document}